\documentclass[12pt,a4paper,openany]{book}
\usepackage{lmodern}
\usepackage[table]{xcolor}
\input{/home/aroquemaurel/cours/includesLaTeX/couleurs.tex}

\usepackage[utf8]{inputenc} \usepackage[T1]{fontenc}
\usepackage[francais]{babel}
\usepackage[top=1.7cm, bottom=1.7cm, left=1.7cm, right=1.7cm]{geometry}
\usepackage{verbatim}
\usepackage[urlbordercolor={1 1 1}, linkbordercolor={1 1 1}, linkcolor=vert1, urlcolor=bleu, colorlinks=true]{hyperref}
\usepackage{tikz} %Vectoriel
\usepackage{listings}
\usepackage{fancyhdr}
\usepackage{multido}
\usepackage{float}
\usepackage{amssymb}
\usepackage{longtable}

\newcommand{\titre}{Conception d'une application Web -- SuperVod}

\newcommand{\pole}{}
\newcommand{\sigle}{bdd}

\newcommand{\semestre}{4}

\input{/home/aroquemaurel/cours/includesLaTeX/listings.tex}
\date{\today}

\makeindex
\lfoot{Université Toulouse III -- Paul Sabatier}
\rfoot{}
%\rfoot{}
\cfoot{}
\makeglossary
\makeatletter
\def\clap#1{\hbox to 0pt{\hss #1\hss}}%
\def\ligne#1{%
\hbox to \hsize{%
\vbox{\centering #1}}}%
\def\haut#1#2#3{%
\hbox to \hsize{%
\rlap{\vtop{\raggedright #1}}%
\hss
\clap{\vtop{\centering #2}}%
\hss
\llap{\vtop{\raggedleft #3}}}}%
\def\bas#1#2#3{%
\hbox to \hsize{%
\rlap{\vbox{\raggedright #1}}%
\hss \clap{\vbox{\centering #2}}%
\hss
\llap{\vbox{\raggedleft #3}}}}%
\def\maketitle{%
\thispagestyle{empty}\vbox to \vsize{%
\haut{}{\@blurb}{}

\vfill
\vspace{1cm}
\begin{flushleft}
\usefont{OT1}{ptm}{m}{n}
\huge \@title
\end{flushleft}
\par
\hrule height 4pt
\par
\begin{flushright}
\usefont{OT1}{phv}{m}{n}
\Large \@author
\par
\end{flushright}
\vspace{1cm}
\vfill
\vfill
\bas{}{\@location, le \@date}{}
}%
\cleardoublepage
}
\def\date#1{\def\@date{#1}}
\def\author#1{\def\@author{#1}}
\def\title#1{\def\@title{#1}}
\def\location#1{\def\@location{#1}}
\def\blurb#1{\def\@blurb{#1}}
\date{\today}
\author{}
\title{}
\location{Amiens}\blurb{}
\makeatother
\title{\titre}
\author{Base de données}

\location{Toulouse}
\blurb{%
Université Toulouse III -- Paul sabatier\\
L2 Informatique\\
\vspace{30px}
\begin{flushleft}Antoine de \bsc{Roquemaurel} (antoine.de-roquemaurel@univ-tlse3.fr)\\ 
	Fabrice \bsc{Valleix} (valleix.fabrice@gmail.com)\\
 Groupe 2.2\end{flushleft}
}%



%\title{Cours \\ \titre}
%\date{\today\\ Semestre \semestre}

%\lhead{Cours: \titre}
%\chead{}
%\rhead{\thepage}

%\lfoot{Université Paul Sabatier Toulouse III}
%\cfoot{\thepage}
%\rfoot{\sigle\semestre}

\pagestyle{fancy}
\renewcommand{\chaptermark}[1]{\markboth{\bsc{\chaptername~\thechapter{} :} #1}{}}
\renewcommand{\sectionmark}[1]{\markright{\thesection{ #1}}}
\renewcommand{\headrulewidth}{0.3pt}
\renewcommand{\footrulewidth}{0.3pt}

\fancyhf{}
\fancyhead[LE]{\leftmark}
\fancyhead[RO]{\rightmark}
\fancyfoot[LE,RO]{--~\thepage~--}
\fancyfoot[LO]{\titre{}}
\fancyfoot[RE]{Antoine de \bsc{Roquemaurel} -- Fabrice \bsc{Valleix}}

%% Cas des premières pages de chapitre
\fancypagestyle{plain}{%
	\fancyhf{}%
	\fancyfoot[L]{\titre{}}
	\fancyfoot[R]{--~\thepage~--}
	\renewcommand{\headrulewidth}{0pt}
	\renewcommand{\footrulewidth}{0.3pt}
}
\makeatletter
\renewcommand*{\lstlistlistingname}{Liste des codes sources}
\renewcommand\listoffigures{%
    \chapter{\listfigurename}%
      \@mkboth{\MakeUppercase\listfigurename}%
              {\MakeUppercase\listfigurename}%
       \@starttoc{lof}%
    }
    \renewcommand\listoftables{%
    \chapter{\listtablename}%
    \@mkboth{\MakeUppercase{\listtablename}}%
            {\MakeUppercase{\listtablename}}%
    \@starttoc{lot}
    }

    \renewcommand\lstlistoflistings{%
    \begingroup
    \chapter{\lstlistlistingname}%
    \parskip\z@\parindent\z@\parfillskip \z@ \@plus 1fil%
    \@starttoc{lol}%
    \endgroup
    }
	\makeatother

\input{/home/aroquemaurel/cours/includesLaTeX/remarquesExempleAttention.tex}
\input{/home/aroquemaurel/cours/includesLaTeX/polices.tex}
\input{/home/aroquemaurel/cours/includesLaTeX/affichageChapitre.tex}
\let\pagebreakORIG\pagebreak
\let\clearpageORIG\clearpage
\let\cleardoublepageORIG\cleardoublepage

\ifx \removepagebreak \undefined
\newcommand{\removepagebreak}{\renewcommand{\pagebreak}{}\renewcommand{\clearpage}{}\renewcommand{\cleardoublepage}{}}
\fi

\ifx \restorepagebreak \undefined
\newcommand{\restorepagebreak}{\renewcommand{\pagebreak}{\pagebreakORIG}\renewcommand{\clearpage}{\clearpageORIG}\renewcommand{\cleardoublepage}{\cleardoublepageORIG}}
\fi
\newcommand{\pfp}{\texttt{pfp}}

\newcommand{\ifp}{\texttt{if}}
\newcommand{\elsep}{\texttt{else}}

\makeatother
\includeonly {
}
\begin{document}
	\setcounter{tocdepth}{2}
	\setcounter{secnumdepth}{3}
	\maketitle
	\chapter*{Avant-propos}
	Ce dossier comporte les différentes de réalisation de l'application Web SuperVod, site permettant de répertorier des séries.

	Il à été conçut par Antoine de \bsc{Roquemaurel} et Fabrice \bsc{Valleix} dans le cadre du module \textit{Systèmes d'Information et Application Web} de la L2 Informatique de l'université Toulouse III -- Paul Sabatier.

	\section*{Tester le projet}
	L'archive que vous avez reçus était organisée comme ceci: 
	\begin{description}
		\item[scriptCreationBd.sql] Contient le script de création de la base de données, le schéma n'a pas été changé, cependant \texttt{MySQL} étant sensible à la
			casse, des erreurs avait lieu avec le script de création qui nous étais fournis, ainsi par convention, toutes nos tables sont en minuscule.
		\item[superVod/] Contient tous les fichiers du Site Web. 
		\item[rapport.pdf] Le présent rapport que vous êtes en train de lire
	\end{description}

	Afin de tester le projet, vous pouvez avoir accès au site web fonctionnel directement en ligne à l'adresse \url{http://dev.joohoo.fr/dev/superVod/}. 
	Il est également possible d'utiliser notre code source avec votre propre serveur web et votre base de données, pour cela les paramétrage des accès à la base de données 
	sont présent dans le fichier \texttt{superVod/database/connect.php}. Toutes les adresses web étant en relatifs, aucun problème ne devrait avoir lieu.

	Ce site Web utilisant des fonctionnalités de \bsc{HTML}5 et \bsc{CSS}3, il est recommandé d'utiliser un navigateur récent. Ce site à été développé sous
	Google Chrome, ainsi l'affichage sera optimal sur ce navigateur, cependant il devrait s'afficher correctement sur les autres.

	Pour se connecter à la partie administration, les identifiants sont:
	\begin{description}
		\item[Login] admin
		\item[Mot de passe] admin
	\end{description}
	\vfill
	\footnotesize Rédigé le \today{} par Antoine de \bsc{Roquemaurel} et Fabrice \bsc{Valleix}
	\tableofcontents
	\chapter{Les besoins de l'application}
	Ce projet consiste en la création d'un site web utilisant les technologies \bsc{HTML}, \bsc{CSS}, \bsc{PHP} et la base de données \bsc{MySQL} permettant
	de répertorier des séries et leurs épisodes associés.  

	Pour cela le schéma de la base de données et un jeu d'essai nous à été fournis afin de pouvoir commencer rapidement le développement du site.
	\section{Affichage des séries et épisodes}
	La possibilité d'afficher dynamiquement les séries et leurs épisodes avec ainsi pour chaque série son titre, son nombre de saison et son nombre
	d'épisodes ainsi que son type, pour chaque épisode d'une série  son titre, sa saison, son numéro d'épisode son année de production, son réalisateur sa
	durée  et sa limite d'âge.

	Étant donné que le site est voué à disposé d énormément de séries différentes, plutôt que de tout afficher sur une page, nous avons souhaités ajouter un
	menu permettant de sélectionner la série. Cf image ?? page ??. % TODO screen accueil
	\section{Recherche de séries et épisodes}
	Le site doit permet d'effectuer des recherches, d'une part des recherches d'épisodes, en fonction d'une partie du titre, ses année de production, son type,
	son age minimum, toutes les séries trouvés grâce à la recherche permette d'afficher les épisodes de la série.

	Mais il doit également être possible de chercher un épisode en fonction de la série concernée, de son année de diffusion, de sa saison et de ses différents
	prix d'achats maximum (Streaming, location ou achat) dans le cas où on souhaite acheter un épisode.

	Des captures d'écrans de recherches sont disponibles image ?? et ?? page ??. % TODO screen recherche
	\section{Ajout de série et épisode}
	% TODO
	\chapter{Organisation du travail d'équipe}
	% TODO
	\section{Un outil de gestion de projet : Redmine}
	% TODO
	\section{Un logiciel de versionnement : Git}
	% TODO
	\chapter{Implémentation}
	% TODO
	\section{Conception}
	% TODO
	\section{Développement}
	% TODO
	\section{Résultats obtenus}
	% TODO
\end{document}

